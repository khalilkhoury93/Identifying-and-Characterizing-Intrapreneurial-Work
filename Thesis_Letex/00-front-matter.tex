\section{Front Matter}\label{front-matter}

\subsection{Title}\label{title}

**Identifying and Characterizing Intrapreneurial Work: A Task-Level
Analysis Using O*NET and Human-AI Collaboration Requirements**

\begin{center}\rule{0.5\linewidth}{0.5pt}\end{center}

\textbf{Author:} {[}Khalil Khoury{]}\\
\textbf{Date:} October 2025 (Revised: October 30, 2025)\\
\textbf{Institution:} {[}University of pecs{]}\\
\textbf{Department:} {[}Faculty of Business and Economics{]}\\
\textbf{Degree:} {[}Master of Science in Business Development{]}\\
\textbf{Supervisor:} {[}Dr. Erika Lázár{]}

\begin{center}\rule{0.5\linewidth}{0.5pt}\end{center}

\subsection{Abstract}\label{abstract}

This study addresses the challenge of objectively identifying and
characterizing intrapreneurial work within organizational contexts by
developing and applying theory-grounded structured criteria to
occupational task descriptions. The research problem centers on three
interrelated questions: where intrapreneurial tasks sit within
occupational structures, what level of human agency they require in an
era of artificial intelligence, and whether they can be reliably
identified through objective methods.

The methodology employs a multi-stage approach using large language
models to classify 844 O\emph{NET task descriptions based on six
theoretical dimensions (opportunity discovery and idea generation; planning, preparation, and advocacy; execution, implementation, and active behavior; innovative and risk-taking behaviors; role-specific managerial tasks; and eco-innovation / green tasks), achieving 92.1\% unanimous agreement across independent
classification runs. These classifications are then analyzed through
O}NET's Importance-Frequency framework and paired with worker and expert
assessments of human agency requirements from the WORKBank database
(Stanford SALT Lab's Future of Work with AI Agents project, 2025).

Key results reveal that intrapreneurial tasks are significantly depleted
in the Core quadrant (high importance, high frequency; OR=0.22,
q\textless0.001) and enriched in Critical (high importance, low
frequency; OR=1.94, q=0.006) and Peripheral (low importance, low
frequency; OR=2.65, q\textless0.001) quadrants. This distribution
demonstrates that innovation work occurs episodically at the edges of
routine occupational practice rather than in routine operational cores.
Human agency analysis shows intrapreneurial tasks concentrate in H3
(Equal human–AI partnership, 33.1\%) and H4 (Human-driven with AI assistance, 22.5\%) bands,
with significantly higher representation in H4 compared to routine tasks
(22.5\% vs 11.2\%). Only 5 intrapreneurial tasks (3.3\%) received worker
H5 ratings (exclusively human), suggesting even highly creative
innovation work is seen as amenable to AI augmentation. Workers
systematically perceive higher task uncertainty than experts across all
quadrants, with the largest gaps in Critical (Δ=+0.58, q\textless0.001)
and Peripheral (Δ=+0.49, q\textless0.001) positions. The willingness to
bear uncertainty index shows that 74\% of workers increase human
agency preferences as uncertainty rises, and 52\% decrease or maintain
automation desires, demonstrating active choice to retain control when
facing unpredictable innovation challenges.

The findings have significant implications for AI augmentation
strategies and organizational design. The concentration of
intrapreneurial work in collaborative (H3) and supervisory (H4) bands
suggests AI systems should prioritize augmentation over automation. The
gap between worker desires for automation support and expert assessments
of technical feasibility is largest in the Peripheral quadrant,
indicating prime opportunities for targeted augmentation tool
development. Worker-expert calibration differences rather than
fundamental disagreement (κ=0.088 categorical, ρ=0.247 continuous)
suggest that adoption strategies must address experiential uncertainty
that workers face in episodic contexts. Organizations must recognize
that strategically important innovation work may appear peripheral from
an occupational perspective, requiring new approaches to task design,
performance evaluation, and human capital development that account for
the episodic nature of intrapreneurial behavior. The decomposition of
intrapreneurial tasks reveals that discovery and planning activities
(63\%+ prevalence) far outnumber execution tasks (41\%), providing
task-level evidence for the knowing-doing gap in organizational
innovation.

\textbf{Keywords:} intrapreneurship, task-level analysis, O*NET, human
agency, AI augmentation, innovation work, occupational structure,
human-AI collaboration

\begin{center}\rule{0.5\linewidth}{0.5pt}\end{center}

\subsection{Table of Contents}\label{table-of-contents}

\textbf{Abstract}
\ldots\ldots\ldots\ldots\ldots\ldots\ldots\ldots\ldots\ldots\ldots\ldots\ldots\ldots\ldots\ldots\ldots\ldots\ldots\ldots\ldots\ldots\ldots\ldots\ldots\ldots\ldots\ldots\ldots..
ii

\textbf{Acknowledgments}
\ldots\ldots\ldots\ldots\ldots\ldots\ldots\ldots\ldots\ldots\ldots\ldots\ldots\ldots\ldots\ldots\ldots\ldots\ldots\ldots\ldots\ldots\ldots\ldots\ldots\ldots..
iii

\textbf{Declaration of Originality}
\ldots\ldots\ldots\ldots\ldots\ldots\ldots\ldots\ldots\ldots\ldots\ldots\ldots\ldots\ldots\ldots\ldots\ldots\ldots\ldots\ldots\ldots\ldots{}
iv

\textbf{List of Figures}
\ldots\ldots\ldots\ldots\ldots\ldots\ldots\ldots\ldots\ldots\ldots\ldots\ldots\ldots\ldots\ldots\ldots\ldots\ldots\ldots\ldots\ldots\ldots\ldots\ldots\ldots\ldots.
v

\textbf{List of Tables}
\ldots\ldots\ldots\ldots\ldots\ldots\ldots\ldots\ldots\ldots\ldots\ldots\ldots\ldots\ldots\ldots\ldots\ldots\ldots\ldots\ldots\ldots\ldots\ldots\ldots\ldots\ldots..
vi

\textbf{Glossary}
\ldots\ldots\ldots\ldots\ldots\ldots\ldots\ldots\ldots\ldots\ldots\ldots\ldots\ldots\ldots\ldots\ldots\ldots\ldots\ldots\ldots\ldots\ldots\ldots\ldots\ldots\ldots\ldots\ldots..
vii

\textbf{Chapter 1: Introduction}
\ldots\ldots\ldots\ldots\ldots\ldots\ldots\ldots\ldots\ldots\ldots\ldots\ldots\ldots\ldots\ldots\ldots\ldots\ldots\ldots\ldots\ldots\ldots\ldots.
1 - 1.1 Problem Motivation and Research Gap
\ldots\ldots\ldots\ldots\ldots\ldots\ldots\ldots\ldots\ldots\ldots\ldots\ldots\ldots\ldots\ldots\ldots{}
1 - 1.2 Research Questions
\ldots\ldots\ldots\ldots\ldots\ldots\ldots\ldots\ldots\ldots\ldots\ldots\ldots\ldots\ldots\ldots\ldots\ldots\ldots\ldots\ldots\ldots\ldots.
4 - 1.3 Research Contributions
\ldots\ldots\ldots\ldots\ldots\ldots\ldots\ldots\ldots\ldots\ldots\ldots\ldots\ldots\ldots\ldots\ldots\ldots\ldots\ldots\ldots\ldots{}
5 - 1.4 Method Overview
\ldots\ldots\ldots\ldots\ldots\ldots\ldots\ldots\ldots\ldots\ldots\ldots\ldots\ldots\ldots\ldots\ldots\ldots\ldots\ldots\ldots\ldots\ldots\ldots{}
8 - 1.5 Thesis Structure
\ldots\ldots\ldots\ldots\ldots\ldots\ldots\ldots\ldots\ldots\ldots\ldots\ldots\ldots\ldots\ldots\ldots\ldots\ldots\ldots\ldots\ldots\ldots..
10

\textbf{Chapter 2: Literature Review}
\ldots\ldots\ldots\ldots\ldots\ldots\ldots\ldots\ldots\ldots\ldots\ldots\ldots\ldots\ldots\ldots\ldots\ldots\ldots\ldots\ldots\ldots{}
12 - 2.1 Introduction
\ldots\ldots\ldots\ldots\ldots\ldots\ldots\ldots\ldots\ldots\ldots\ldots\ldots\ldots\ldots\ldots\ldots\ldots\ldots\ldots\ldots\ldots\ldots\ldots\ldots.
12 - 2.2 Theoretical Foundations of Intrapreneurship
\ldots\ldots\ldots\ldots\ldots\ldots\ldots\ldots\ldots\ldots\ldots\ldots\ldots\ldots{}
15 - 2.3 Task Structure and Occupational Positioning
\ldots\ldots\ldots\ldots\ldots\ldots\ldots\ldots\ldots\ldots\ldots\ldots\ldots\ldots{}
22 - 2.4 Human Agency, Autonomy, and AI-Augmented Work
\ldots\ldots\ldots\ldots\ldots\ldots\ldots\ldots\ldots\ldots\ldots\ldots\ldots.
28 - 2.5 Measurement and Operationalization
\ldots\ldots\ldots\ldots\ldots\ldots\ldots\ldots\ldots\ldots\ldots\ldots\ldots\ldots\ldots\ldots\ldots{}
35 - 2.6 Organizational and Individual Determinants
\ldots\ldots\ldots\ldots\ldots\ldots\ldots\ldots\ldots\ldots\ldots\ldots\ldots\ldots.
40 - 2.7 Synthesis and Transition to Analysis
\ldots\ldots\ldots\ldots\ldots\ldots\ldots\ldots\ldots\ldots\ldots\ldots..
45

\textbf{Chapter 3: Methods and Data}
\ldots\ldots\ldots\ldots\ldots\ldots\ldots\ldots\ldots\ldots\ldots\ldots\ldots\ldots\ldots\ldots\ldots\ldots\ldots\ldots\ldots\ldots.
48 - 3.1 Data Provenance
\ldots\ldots\ldots\ldots\ldots\ldots\ldots\ldots\ldots\ldots\ldots\ldots\ldots\ldots\ldots\ldots\ldots\ldots\ldots\ldots\ldots\ldots\ldots\ldots{}
48 - 3.2 Samples and Scopes
\ldots\ldots\ldots\ldots\ldots\ldots\ldots\ldots\ldots\ldots\ldots\ldots\ldots\ldots\ldots\ldots\ldots\ldots\ldots\ldots\ldots\ldots\ldots{}
50 - 3.3 Label Adjudication Process
\ldots\ldots\ldots\ldots\ldots\ldots\ldots\ldots\ldots\ldots\ldots\ldots\ldots\ldots\ldots\ldots\ldots\ldots\ldots\ldots{}
52 - 3.4 Importance-Frequency Quadrant Construction
\ldots\ldots\ldots\ldots\ldots\ldots\ldots\ldots\ldots\ldots\ldots\ldots\ldots\ldots.
55 - 3.5 Human Agency Scale Framework
\ldots\ldots\ldots\ldots\ldots\ldots\ldots\ldots\ldots\ldots\ldots\ldots\ldots\ldots\ldots\ldots\ldots\ldots\ldots.
57 - 3.6 Statistical Procedures
\ldots\ldots\ldots\ldots\ldots\ldots\ldots\ldots\ldots\ldots\ldots\ldots\ldots\ldots\ldots\ldots\ldots\ldots\ldots\ldots\ldots..
59 - 3.7 Ethics and Data Handling
\ldots\ldots\ldots\ldots\ldots\ldots\ldots\ldots\ldots\ldots\ldots\ldots\ldots\ldots\ldots\ldots\ldots\ldots\ldots\ldots..
61 - 3.8 Reproducibility Provisions
\ldots\ldots\ldots\ldots\ldots\ldots\ldots\ldots\ldots\ldots\ldots\ldots\ldots\ldots\ldots\ldots..
62 - 3.9 Secondary Categorization of Intrapreneurial Tasks
\ldots\ldots\ldots\ldots\ldots\ldots\ldots\ldots\ldots\ldots\ldots\ldots\ldots\ldots\ldots\ldots\ldots{}
64

\textbf{Chapter 4: Results and Analysis}
\ldots\ldots\ldots\ldots\ldots\ldots\ldots\ldots\ldots\ldots\ldots\ldots\ldots\ldots\ldots\ldots\ldots\ldots\ldots\ldots..
64 - 4.1 Classification Results
\ldots\ldots\ldots\ldots\ldots\ldots\ldots\ldots\ldots\ldots\ldots\ldots\ldots\ldots\ldots\ldots\ldots\ldots\ldots\ldots\ldots..
64 - 4.2 Importance-Frequency Distribution
\ldots\ldots\ldots\ldots\ldots\ldots\ldots\ldots\ldots\ldots\ldots\ldots\ldots\ldots\ldots\ldots\ldots..
66 - 4.3 Human Agency Requirements
\ldots\ldots\ldots\ldots\ldots\ldots\ldots\ldots\ldots\ldots\ldots\ldots\ldots\ldots\ldots\ldots\ldots\ldots\ldots\ldots.
70 - 4.4 Perceived Uncertainty Analysis
\ldots\ldots\ldots\ldots\ldots\ldots\ldots\ldots\ldots\ldots\ldots\ldots\ldots\ldots\ldots\ldots\ldots\ldots..
73 - 4.5 Work Activity Signatures
\ldots\ldots\ldots\ldots\ldots\ldots\ldots\ldots\ldots\ldots\ldots\ldots\ldots\ldots\ldots\ldots\ldots\ldots\ldots\ldots\ldots{}
75 - 4.6 Illustrative Examples
\ldots\ldots\ldots\ldots\ldots\ldots\ldots\ldots\ldots\ldots\ldots\ldots\ldots\ldots\ldots\ldots\ldots\ldots\ldots\ldots..
77 - 4.7 Internal Structure of Intrapreneurial Tasks
\ldots\ldots\ldots\ldots\ldots\ldots\ldots\ldots\ldots\ldots\ldots\ldots\ldots\ldots\ldots\ldots\ldots\ldots\ldots\ldots\ldots..
79 - 4.8 Robustness Across Threshold Specifications
\ldots\ldots\ldots\ldots\ldots\ldots\ldots\ldots\ldots\ldots\ldots\ldots\ldots\ldots\ldots\ldots..
81 - 4.9 Terminology
\ldots\ldots\ldots\ldots\ldots\ldots\ldots\ldots\ldots\ldots\ldots\ldots\ldots\ldots\ldots\ldots\ldots..
84 - 4.10 Summary and Implications
\ldots\ldots\ldots\ldots\ldots\ldots\ldots\ldots\ldots\ldots\ldots\ldots\ldots\ldots\ldots\ldots\ldots\ldots\ldots\ldots{}
85

\textbf{Chapter 5: Discussion}
\ldots\ldots\ldots\ldots\ldots\ldots\ldots\ldots\ldots\ldots\ldots\ldots\ldots\ldots\ldots\ldots\ldots\ldots\ldots\ldots\ldots\ldots\ldots\ldots.
82 - 5.1 Interpreting the Core Findings
\ldots\ldots\ldots\ldots\ldots\ldots\ldots\ldots\ldots\ldots\ldots\ldots\ldots\ldots\ldots\ldots\ldots\ldots..
82 - 5.2 Theoretical Implications
\ldots\ldots\ldots\ldots\ldots\ldots\ldots\ldots\ldots\ldots\ldots\ldots\ldots\ldots\ldots\ldots\ldots\ldots\ldots\ldots..
85 - 5.3 Organizational and AI Design Implications
\ldots\ldots\ldots\ldots\ldots\ldots\ldots\ldots\ldots\ldots\ldots\ldots\ldots\ldots..
90 - 5.4 Limitations
\ldots\ldots\ldots\ldots\ldots\ldots\ldots\ldots\ldots\ldots\ldots\ldots\ldots\ldots\ldots\ldots\ldots\ldots\ldots\ldots\ldots\ldots\ldots\ldots\ldots.
95 - 5.5 Future Research Directions
\ldots\ldots\ldots\ldots\ldots\ldots\ldots\ldots\ldots\ldots\ldots\ldots\ldots\ldots\ldots\ldots\ldots\ldots\ldots\ldots{}
98

\textbf{Chapter 6: Conclusion}
\ldots\ldots\ldots\ldots\ldots\ldots\ldots\ldots\ldots\ldots\ldots\ldots\ldots\ldots\ldots\ldots\ldots\ldots\ldots\ldots\ldots\ldots\ldots\ldots.
98 - 6.1 Answering the Research Questions
\ldots\ldots\ldots\ldots\ldots\ldots\ldots\ldots\ldots\ldots\ldots\ldots\ldots\ldots\ldots\ldots\ldots..
98 - 6.2 Contributions and Practical Implications
\ldots\ldots\ldots\ldots\ldots\ldots\ldots\ldots\ldots\ldots\ldots\ldots\ldots\ldots..
101 - 6.3 Augmentation Over Automation
\ldots\ldots\ldots\ldots\ldots\ldots\ldots\ldots\ldots\ldots\ldots\ldots\ldots\ldots\ldots\ldots\ldots\ldots\ldots{}
103 - 6.4 Closing Synthesis
\ldots\ldots\ldots\ldots\ldots\ldots\ldots\ldots\ldots\ldots\ldots\ldots\ldots\ldots\ldots\ldots\ldots\ldots\ldots\ldots\ldots\ldots..
105

\textbf{References}
\ldots\ldots\ldots\ldots\ldots\ldots\ldots\ldots\ldots\ldots\ldots\ldots\ldots\ldots\ldots\ldots\ldots\ldots\ldots\ldots\ldots\ldots\ldots\ldots\ldots\ldots\ldots\ldots{}
107

\textbf{Appendices}
\ldots\ldots\ldots\ldots\ldots\ldots\ldots\ldots\ldots\ldots\ldots\ldots\ldots\ldots\ldots\ldots\ldots\ldots\ldots\ldots\ldots\ldots\ldots\ldots\ldots\ldots\ldots\ldots{}
120 - Appendix A: Survey Instruments
\ldots\ldots\ldots\ldots\ldots\ldots\ldots\ldots\ldots\ldots\ldots\ldots\ldots\ldots\ldots\ldots\ldots\ldots\ldots.
120 - Appendix B: Full LLM Classification Prompt
\ldots\ldots\ldots\ldots\ldots\ldots\ldots\ldots\ldots\ldots\ldots\ldots\ldots\ldots\ldots{}
122 - Appendix C: Robustness Tables and Additional Figures
\ldots\ldots\ldots\ldots\ldots\ldots\ldots\ldots\ldots\ldots\ldots. 125
- Appendix D: Scripts, Paths, and Reproducibility Notes
\ldots\ldots\ldots\ldots\ldots\ldots\ldots\ldots\ldots\ldots\ldots{} 128
- Appendix E: Data Dictionary and Sample Details
\ldots\ldots\ldots\ldots\ldots\ldots\ldots\ldots\ldots\ldots\ldots\ldots\ldots..
131 - Appendix F: Extended Analysis of Intrapreneurial Task Structure
\ldots\ldots\ldots\ldots\ldots\ldots\ldots.. 135

\begin{center}\rule{0.5\linewidth}{0.5pt}\end{center}

\subsection{List of Figures}\label{list-of-figures}

\textbf{Figure 4.1:} Distribution of intrapreneurial tasks across
Importance-Frequency quadrants \ldots\ldots\ldots\ldots.. 66

\textbf{Figure 4.2:} Robustness of IF quadrant prevalence across
thresholds
\ldots\ldots\ldots\ldots\ldots\ldots\ldots\ldots\ldots\ldots.. 68

\textbf{Figure 4.3:} Human Agency Scale band distribution for
intrapreneurial vs non-intrapreneurial tasks \ldots{} 70

\textbf{Figure 4.4:} Worker versus expert Human Agency Scale ratings
scatter plot \ldots\ldots\ldots\ldots\ldots\ldots\ldots. 72

\textbf{Figure 4.5:} Worker-expert uncertainty gaps by IF quadrant for
intrapreneurial tasks \ldots\ldots\ldots\ldots{} 74

\textbf{Figure 4.6:} Willingness to Bear Uncertainty (WBU) distribution
\ldots\ldots\ldots\ldots\ldots\ldots\ldots\ldots\ldots\ldots\ldots\ldots\ldots.
76

\textbf{Figure 3.1:} Intrapreneurship classification decision flowchart
(Criteria I-VI) \ldots\ldots\ldots\ldots\ldots. 54

\textbf{Figure 4.7:} Category prevalence within intrapreneurial tasks
\ldots\ldots\ldots\ldots\ldots\ldots\ldots\ldots\ldots\ldots\ldots. 82

\textbf{Figure 4.8:} Intrapreneurship phenotypes and their prevalence
\ldots\ldots\ldots\ldots\ldots\ldots\ldots\ldots\ldots\ldots\ldots. 83

\begin{center}\rule{0.5\linewidth}{0.5pt}\end{center}

\textbf{Appendix C Figures:}

\textbf{Figure C.X:} Worker-expert HAS band agreement matrix (confusion
matrix) \ldots\ldots\ldots\ldots\ldots\ldots\ldots.. 126

\begin{center}\rule{0.5\linewidth}{0.5pt}\end{center}

\textbf{Appendix F Figures:}

\textbf{Figure F.1:} Top category associations (positive and negative)
\ldots\ldots\ldots\ldots\ldots\ldots\ldots\ldots\ldots\ldots.. 136

\textbf{Figure F.2:} Category co-occurrence heatmap
\ldots\ldots\ldots\ldots\ldots\ldots\ldots\ldots\ldots\ldots\ldots\ldots\ldots\ldots\ldots\ldots\ldots.
137

\textbf{Figure F.3:} Phenotype analysis summary dashboard
\ldots\ldots\ldots\ldots\ldots\ldots\ldots\ldots\ldots\ldots\ldots\ldots\ldots\ldots\ldots.
138

\textbf{Figure F.4:} Phenotype overlap matrix
\ldots\ldots\ldots\ldots\ldots\ldots\ldots\ldots\ldots\ldots\ldots\ldots\ldots\ldots\ldots\ldots\ldots\ldots\ldots.
139

\begin{center}\rule{0.5\linewidth}{0.5pt}\end{center}

\subsection{List of Tables}\label{list-of-tables}

\textbf{Table 3.1:} Sample sizes and scope specifications
\ldots\ldots\ldots\ldots\ldots\ldots\ldots\ldots\ldots\ldots\ldots\ldots\ldots\ldots{}
51

\textbf{Table 4.1:} Classification outcomes and voting consistency
\ldots\ldots\ldots\ldots\ldots\ldots\ldots\ldots\ldots\ldots\ldots. 65

\textbf{Table 4.2:} Intrapreneurial task prevalence by IF quadrant
(thresholds 4.0/4.0) \ldots\ldots\ldots\ldots\ldots{} 67

\textbf{Table 4.3:} Directional consistency across threshold schemes
\ldots\ldots\ldots\ldots\ldots\ldots\ldots\ldots\ldots\ldots. 69

\textbf{Table 4.4:} HAS band distribution: intrapreneurial vs
non-intrapreneurial tasks \ldots\ldots\ldots\ldots.. 71

\textbf{Table 4.5:} Worker vs expert perceived uncertainty for
intrapreneurial tasks \ldots\ldots\ldots\ldots\ldots. 74

\textbf{Table 4.7:} Mean WBU by IF quadrant for intrapreneurial tasks
\ldots\ldots\ldots\ldots\ldots\ldots\ldots\ldots\ldots.. 77

\textbf{Table 4.8:} Top Work Activities enrichments for intrapreneurial
tasks \ldots\ldots\ldots\ldots\ldots\ldots\ldots. 78

\textbf{Table 4.9:} Representative intrapreneurial tasks by IF quadrant
\ldots\ldots\ldots\ldots\ldots\ldots\ldots\ldots\ldots{} 80



\begin{center}\rule{0.5\linewidth}{0.5pt}\end{center}

\textbf{Appendix E Tables:}

\textbf{Table E.1:} Tasks excluded from HAS-complete subset
\ldots\ldots\ldots\ldots\ldots\ldots\ldots\ldots\ldots\ldots\ldots\ldots\ldots.
132

\begin{center}\rule{0.5\linewidth}{0.5pt}\end{center}

\textbf{Appendix F Tables:}

\textbf{Table F.1:} Ambiguous task detail
\ldots\ldots\ldots\ldots\ldots\ldots\ldots\ldots\ldots\ldots\ldots\ldots\ldots\ldots\ldots\ldots\ldots\ldots\ldots.
135

\begin{center}\rule{0.5\linewidth}{0.5pt}\end{center}

\subsection{Acknowledgments}\label{acknowledgments}

I would like to express my deepest gratitude to my supervisor,
{[}Supervisor Name{]}, for their invaluable guidance, continuous
support, and patience throughout this research journey. Their expertise
and insights have been instrumental in shaping this work.

I extend my sincere thanks to the Stanford SALT Lab team for providing
access to the WORKBank database, which formed the data foundation
of this study. Special appreciation goes to the researchers who
contributed to the development of the Human Agency Scale framework.

I am grateful to my committee members, {[}Committee Member Names{]}, for
their constructive feedback and thoughtful questions that strengthened
this research.

I thank my colleagues in {[}Department/Lab Name{]} for stimulating
discussions and collaborative support. Their diverse perspectives
enriched my understanding of the research domain.

Finally, I am deeply grateful to my family and friends for their
unwavering encouragement and understanding throughout this academic
endeavor. Their support has been my constant source of strength.

{[}Additional acknowledgments for funding sources, if applicable{]}

\begin{center}\rule{0.5\linewidth}{0.5pt}\end{center}

\subsection{Declaration of
Originality}\label{declaration-of-originality}

I hereby declare that this thesis is my own original work and that it
has not been submitted, in whole or in part, for any other degree or
qualification at this or any other university or institution.

All sources of information have been properly acknowledged and
referenced according to academic conventions. Where I have drawn upon
the work, ideas, or words of others, this has been acknowledged in the
text and in the list of references.

The analyses presented in this thesis were conducted using
publicly available data from the WORKBank database and O*NET, with
appropriate permissions and in accordance with ethical guidelines. The
computational analyses and interpretations are my own work.

I confirm that this thesis does not exceed the prescribed word limit for
the degree of {[}Master of Science / Doctor of Philosophy{]} in {[}Your
Department{]}.

\textbf{Signature:}
\_\_\_\_\_\_\_\_\_\_\_\_\_\_\_\_\_\_\_\_\_\_\_\_\_\_\_\_\_

\textbf{Name:} {[}Your Full Name{]}

\textbf{Date:} {[}Submission Date{]}









