\section{Conclusion}

\subsection{Research Questions Resolved}

Our investigation began with three interconnected questions about the nature and positioning of intrapreneurial work. The evidence suggests a coherent pattern: innovation work tends to operate at occupational margins (RQ1), appearing in Critical and Peripheral quadrants where decisions are episodic rather than routine. In the Critical quadrant these episodes are high-uncertainty/high-stakes; in the Peripheral quadrant they are rated as low importance in O*NET terms yet can still be strategically significant when their effects materialize with lags or outside standard performance metrics. This peripheral positioning corresponds with elevated human agency requirements (RQ2): both workers and experts assign these tasks to H3 (Equal human–AI partnership) and H4 (Human-driven with AI assistance) collaborative bands rather than automation zones. The classification methodology (RQ3) appears stable, with 92.1\% agreement, suggesting that theoretical constructs can be operationalized usefully at scale in this context. Together, these findings indicate that innovation work is structurally distinct from routine operations and may require different management approaches and augmentation strategies.

\textbf{RQ1: How are intrapreneurial tasks distributed across O*NET Importance × Frequency quadrants?} Intrapreneurial tasks are significantly depleted in the Core quadrant (OR = 0.22, q < 0.001) where routine operational work dominates. Instead, they concentrate in Critical (OR = 1.94, q = 0.006) and Peripheral (OR = 2.65, q < 0.001) positions, consistent with the idea that strategic innovation work occurs episodically at occupational peripheries rather than within routine centers. This pattern holds across multiple threshold specifications, suggesting that results are unlikely to be solely due to threshold choices (see Section~\ref{sec:results} for full statistics, Section~\ref{sec:ch5-integrated-model} for interpretation).

\textbf{RQ2: How do worker and expert assessments of human agency and automation potential differ for intrapreneurial versus non-intrapreneurial tasks?} The human agency analysis reinforces the peripheral positioning through automation resistance patterns. Intrapreneurial tasks cluster in collaborative bands H3 (33.1\%) and H4 (22.5\%), with only 5 tasks (3.3\%) receiving exclusively human H5 ratings from workers. Workers perceive systematically higher uncertainty than experts, especially in Critical (Δ = +0.58) and Peripheral (Δ = +0.49) quadrants where q < 0.001. When uncertainty rises, 74\% of workers report increasing their human-agency preferences, which is consistent with a greater stated willingness to bear uncertainty rather than delegate to machines. While categorical agreement between workers and experts remains modest (κ = 0.088), continuous correlation shows meaningful alignment (ρ = 0.247). These patterns are consistent with the idea that innovation work requires human judgment precisely where uncertainty is highest (Sections~\ref{sec:if-distribution}–\ref{sec:human-agency}; see also Section~\ref{sec:ch5-agency-affordance}).

\textbf{RQ3: Can intrapreneurial tasks be identified reliably using theory-grounded criteria applied to task descriptions?}

We encode six theory‑grounded criteria (I–VI; Sections~\ref{sec:ch2-foundations} and~\ref{sec:ch3-criteria-dev}) into a structured LLM prompt (Appendix~\ref{app:B1}) and classify each O*NET task via three independent runs with majority voting, yielding reproducible labels while preserving theoretical fidelity. The category tags used in secondary analyses (I–IV, VI, and V.A/V.B/V.C) come directly from the model’s \texttt{matched\_categories} output. Work Activity validation supports construct validity: nearly five‑fold enrichment in “Thinking Creatively” (OR=4.84) and severe depletion in “Documenting/Recording Information” (OR=0.13). Methodology detailed in Section~\ref{sec:ch3-label-adjudication}; classification results in Section~\ref{sec:results}.

\subsection{Contributions and Practical Implications}

Our research makes three primary contributions with direct practical applications. First, it provides evidence consistent with the idea that innovation work operates according to different organizing principles than routine work, occurring episodically at occupational peripheries rather than continuously at occupational cores. Organizations can consider redesigning performance evaluation and career development systems to recognize and reward irregular but strategically significant innovation contributions that traditional metrics may undervalue.

Second, the study suggests that intrapreneurial tasks require collaborative and supervisory human–AI interaction rather than full automation, with workers reporting a greater willingness to bear uncertainty rather than delegate to machines. This concentration in H3–H4 bands suggests that AI system design is likely to be more effective when prioritizing augmentation architectures that amplify human creativity and judgment while providing computational support for pattern recognition and analysis, and that such systems are likely to be more effective when they preserve meaningful human decision authority even while offering sophisticated analytical capabilities.

Third, the research provides a reproducible methodology for identifying innovation-oriented work at scale, enabling evidence-based workforce planning and development. Human resource professionals can apply these classification criteria to audit organizational task portfolios, identify innovation capacity gaps, and design targeted interventions. The high agreement rates suggest that subjective concepts like intrapreneurship can be operationalized usefully for practical application in this context.

\textbf{Fourth}, by decomposing intrapreneurial tasks into theory‑aligned components using the LLM‑derived category system (opportunity discovery, planning/preparation/advocacy, execution, innovation, and managerial enabling; see Sections~\ref{sec:ch3-category-framework}–\ref{sec:ch3-phenotype-derivation}) and showing that discovery/planning forms dominate (63\%+ prevalence) while execution lags (41\%), we provide task‑level evidence consistent with the knowing–doing gap in organizational innovation (Pfeffer \& Sutton, 2000). This decomposition connects large‑scale occupational data to established intrapreneurship process models (Burgelman, 1983; Kuratko et al., 2005) and offers task‑level evidence consistent with Antoncic and Hisrich's (2001) multidimensional construct. The finding that managerial phenotypes account for 42\% of intrapreneurial tasks is consistent with the view that innovation work is embedded in organizational systems requiring autonomy, resources, and championing, not merely individual creativity.

The systematic gap between worker and expert uncertainty perceptions has immediate implications for technology adoption strategies. In Critical and Peripheral quadrants where gaps are largest, organizations can involve workers in co-designing augmentation tools that address experienced complexity rather than relying solely on expert technical assessments. The positive relationship between uncertainty and human agency preferences suggests that workers may benefit from confidence-building and capability development alongside technological tools.

For organizational design, the findings challenge conventional wisdom about job crafting and role definition. Rather than concentrating innovation responsibilities in specialized positions, organizations might foster intrapreneurship by creating space for episodic innovation work within diverse roles. This could involve: temporal allocation for exploration beyond routine duties; recognition systems that value peripheral contributions; and cross-functional structures that enable boundary-spanning innovation work.

\subsection{Augmentation Over Automation}

The evidence points toward augmentation rather than automation as a more appropriate framework for supporting intrapreneurial work. The scarcity of worker-rated H5 tasks (only 5, or 3.3\%) might initially suggest widespread automation potential, but the concentration in H3–H4 bands reveals a more nuanced reality: innovation work requires human–AI collaboration where humans retain agency and judgment while leveraging computational capabilities.

A particularly large opportunity for augmentation tool development lies in the Peripheral quadrant, where worker desires for support exceed expert assessments of technical feasibility. These episodic, cognitively demanding tasks may benefit from scaffolding tools that support execution without removing human control: just-in-time guidance systems, context-resurrection tools, and creativity support platforms that amplify rather than substitute human capabilities.

In the Critical quadrant, expert-assessed capability often exceeds worker acceptance (experts place ~50\% of intrapreneurial tasks in H1/H2 vs ~36\% for workers; expert capability mean \(\approx\) 3.12). In the Core intrapreneurial slice, workers and experts assign similar automation-prone shares (~35.5\% each) and expert capability is moderate (\(\approx\) 3.09), so evidence for a capability-over-acceptance gap is weaker. In both settings, success is likely to require change management that addresses identity and meaning alongside efficiency; adoption strategies are likely to be more effective when they emphasize capability enhancement and co-design rather than efficiency-driven substitution.

\subsection{Closing Synthesis}

Our research suggests intrapreneurial work's paradoxical positioning (Section~\ref{sec:ch5-integrated-model}): strategically vital yet occupationally peripheral, episodic yet mission-critical, requiring elevated human agency yet benefiting from computational support. By examining 844 occupational tasks through multiple lenses (theoretical classification, importance-frequency positioning, and human agency requirements) we provide evidence consistent with the idea that innovation work operates according to different organizing principles than operational work. It emerges irregularly rather than continuously, clusters at peripheries rather than cores, and demands human judgment even as it benefits from AI augmentation.

The measurement framework developed here suggests that intrapreneurship can be identified objectively and at scale in this context, moving beyond reliance on self-reports toward reproducible classification. The 92.1\% agreement rate across independent assessments indicates that theoretical constructs can be operationalized usefully for practical application here. This methodological contribution also illustrates how complex organizational phenomena can be studied systematically using large-scale occupational data.

The implications for organizational design may be substantial. If innovation work tends to occur at occupational peripheries, then organizations seeking to foster intrapreneurship may need to look beyond job descriptions to the margins where experimentation happens. Performance systems that reward consistent execution of core tasks may systematically undervalue employees who excel at episodic innovation. Career paths that emphasize deepening expertise in routine domains may miss individuals skilled at boundary-spanning creativity.

Organizations might consider implementing concrete mechanisms to recognize and reward episodic innovation work: (1) episodic innovation logs where employees document irregular innovation contributions with timestamps and outcomes, enabling retrospective evaluation beyond daily performance metrics; (2) semi-annual innovation bonuses distributed 6 months after project completion to capture lagged value creation that traditional quarterly reviews miss; and (3) cross-unit sponsorship credits where employees receive formal recognition for helping innovation initiatives in other departments, acknowledging the boundary-spanning nature of intrapreneurial work. These mechanisms are designed to address the temporal mismatch between episodic innovation contributions and continuous performance evaluation cycles.

For the future of work, our findings suggest that human creativity and judgment remain essential even as AI capabilities advance. The concentration of intrapreneurial tasks in collaborative and supervisory bands indicates that the human role in innovation will shift but not disappear. Workers will increasingly partner with AI systems that handle routine variations while humans navigate uncertainty, exercise judgment, and pursue opportunities in less-defined problem spaces.

Our findings can be summarized by a simple heuristic for innovation work: innovation value depends jointly on peripheral positioning, human agency, and uncertainty tolerance. Innovation tends to occur at occupational peripheries (rather than in routine cores), requires human agency (rather than full automation), and involves inherent uncertainty (rather than predictable outcomes). Organizations that support episodic work at the edges, preserve human judgment in uncertain domains, and augment rather than automate creative tasks may be better positioned to capture the innovation premium in an AI-augmented economy.

The episodic nature of innovation work presents both challenge and opportunity. The challenge lies in maintaining readiness for irregular creative demands while managing routine responsibilities. The opportunity lies in recognizing that innovation is unlikely to be fully routinized or automated. It often requires human agents willing to bear uncertainty and pursue possibilities that others overlook. By understanding where and how intrapreneurial work occurs within occupational structures, organizations can better design systems that nurture human creativity in partnership with artificial intelligence. A practical question is how humans and machines can innovate together.
