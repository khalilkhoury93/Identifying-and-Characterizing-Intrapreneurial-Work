\section{Literature Review}
\label{sec:ch2}

\subsection{Introduction}
\label{sec:ch2-intro}

\noindent This chapter reviews scholarship that informs our task-level operationalization of intrapreneurship. Rather than revisiting the problem statement and research questions (see Chapter~1), we focus on four threads: (i) conceptual lenses that yield actionable criteria; (ii) how innovation tasks are positioned within occupations; (iii) human--AI agency requirements; and (iv) measurement beyond self-report.

\noindent Roadmap. Section~\ref{sec:ch2-foundations} distills theoretical foundations into criteria; Section~\ref{sec:ch2-task-structure} situates tasks using O*NET and IF quadrants; Section~\ref{sec:ch2-agency} reviews human agency and augmentation; Section~\ref{sec:ch2-measurement} surveys measurement options; Section~\ref{sec:ch2-determinants} summarizes organizational/individual determinants.

\subsection{Theoretical Foundations of Intrapreneurship}
\label{sec:ch2-foundations}

\subsubsection{Definitions and Conceptual Development}

Research on entrepreneurship inside established organizations gained momentum in the early 1980s (Burgelman, 1983; Miller, 1983). Building on this work, Pinchot (1985) introduced and popularized the term “intrapreneur” to describe employees who act as entrepreneurs within their organizations, taking hands-on responsibility for creating innovation of any kind using corporate resources. His initial conceptualization emphasizes intrapreneurs who use corporate resources to create new products, processes, services, and internal ventures, often framed as “intraprises” within large corporations rather than independent startups. This focus reflected the strategic priorities of large corporations seeking to maintain competitiveness through internal innovation.

Miller (1983) argued that firms can exhibit entrepreneurial behavior via innovativeness, risk taking, and proactiveness. This tripartite framework became foundational for subsequent research, though scholars have debated whether the core dimensions of entrepreneurial orientation should be treated as a single, unified construct or as distinct dimensions with potentially different effects (Lumpkin and Dess, 1996). The distinction between innovation (introducing new products or processes), risk-taking (venturing into uncertain markets or technologies), and proactivity (anticipating future needs and opportunities) remains central to contemporary definitions of entrepreneurial orientation and intrapreneurship (Miller, 1983; Lumpkin and Dess, 1996; Blanka, 2018; Neessen et al., 2018).

Zahra (1993) argues that the Covin–Slevin entrepreneurial posture model largely emphasizes the intensity of entrepreneurial behavior and overlooks other key dimensions: the formality of entrepreneurial activities (formal “induced” vs. informal “autonomous”), the type of internal venture (e.g., administrative, opportunistic, imitative, acquisitive, incubative), and the duration of entrepreneurial efforts. Examining only one of these dimensions “captures a ‘slice’” of firm-level entrepreneurship and can lead to underestimation of its contribution to performance (Zahra, 1993).

The shift in focus from corporate entrepreneurship to intrapreneurship highlights that innovation can stem from employees’ efforts, not only from top‑down strategic directives (De Jong and Wennekers, 2008; Parker, 2011). In this view, intrapreneurship refers to entrepreneurial behavior by employees within existing organizations (De Jong and Wennekers, 2008). This form of entrepreneurship is distinguishable from independent entrepreneurship because ventures occur within an organization and draw on its resources. Parker (2011) further explains that pursuing new venture development inside a company entails different risks and rewards than independent entrepreneurship because employees leverage organizational resources and structures. Even though both contain overlapping innovative work behavior, Åmo (2010) argues that intrapreneurship, situated within an organization, involves more self‑initiated and self‑contained actions, whereas corporate entrepreneurship emphasizes the organization’s actions. Comparative evidence also sheds light on distinctions and similarities between intrapreneurs, independent entrepreneurs, and founders of spin‑off companies (Bager, Ottosson, and Schott, 2010).

Following Zahra's (1993) critique of unidimensional models, we treat entrepreneurship as a multi-level phenomenon that can occur at corporate, SBU, functional, and, even more granularly, task levels; failing to model these levels can produce misleading conclusions about where entrepreneurial work actually resides. Longitudinal evidence indicates that corporate entrepreneurship is associated with modest performance improvements initially that become more pronounced over multi‑year horizons, with the effect appearing strongest in hostile environments where opportunities are scarce, suggesting that the payoff to entrepreneurial behavior can be both lagged and context‑dependent (Zahra and Covin, 1995).

Contemporary frameworks increasingly emphasize behavioral manifestations over personality traits or intentions. Neessen et al. (2018) synthesize the literature to identify five core intrapreneurial behaviors: (1) innovativeness, (2) proactiveness, (3) risk-taking, (4) opportunity recognition and exploitation, and (5) internal/external networking. This behavioral focus enables more precise measurement and acknowledges that intrapreneurship may be situationally activated rather than a stable individual characteristic. Related behaviors often discussed in adjacent work include resource acquisition and orchestration, perseverance/commitment, and strategic renewal; in this review we treat these as determinants or implementation roles (see Section~\ref{sec:ch2-determinants}) rather than core behaviors.

The emergence of social intrapreneurship as a distinct phenomenon extends the conceptual boundaries of traditional intrapreneurship. Geradts and Alt (2022) argue that entrepreneurial action aimed at solving social problems within established organizations faces unique challenges and opportunities compared to commercial intrapreneurship. By synthesizing the de novo social entrepreneurship and intrapreneurship literatures, they argue that social intrapreneurs need to navigate the three core elements of entrepreneurial action (innovation, resource allocation, and uncertainty) within the additional constraints of social value creation. Their framework suggests that established organizations' structural constraints, elaborate rules, and management frameworks pose particular obstacles to social intrapreneurship, while simultaneously offering resources and scale advantages unavailable to independent social entrepreneurs. This conceptualization enriches our understanding of intrapreneurial diversity and highlights the importance of considering mission orientation alongside behavioral dimensions.

Complementing these foundations, Frese and Gielnik (2022) synthesize an action-oriented process view (action theory) that links cognition, motivation, and entrepreneurial action across time. Their action theory lens frames intrapreneurship as iterative cycles of goal setting, planning, execution, and feedback embedded in organizational contexts, an approach we adopt to connect task design with observed innovative behaviors. This aligns with an intention-based action view in which goal setting, planning, and feedback cycles channel attention and behavior toward venture creation (Bird, 1988).

\subsubsection{Theoretical Criteria for Identifying Intrapreneurial Behavior}

These criteria capture opportunity discovery, planning, preparation, and advocacy, execution and implementation under uncertainty, innovative and risk-taking behaviors, role-specific managerial enabling, and eco-innovation / environmental performance.

\textbf{Opportunity Discovery and Recognition.} Intrapreneurial tasks involve identifying unmet needs, market gaps, or process inefficiencies that present opportunities for value creation (McMullen and Shepherd, 2006). This requires environmental scanning, pattern recognition, and the ability to envision alternative futures. Unlike routine problem-solving, opportunity recognition involves identifying possibilities that others have overlooked or dismissed. The nature of opportunity recognition itself remains contested. Alvarez and Barney (2007) distinguish between discovery theory, where opportunities exist independently waiting to be found, and creation theory, where opportunities emerge through entrepreneurial action itself. This theoretical distinction has implications for intrapreneurship: discovery-oriented intrapreneurs focus on search and evaluation processes, while creation-oriented intrapreneurs emphasize iterative experimentation and the construction of new products, services, or markets through action.

\textbf{Planning, Preparation, and Advocacy.} Intrapreneurs must acquire and orchestrate resources (financial, human, technological) often without formal authority or dedicated budgets. This involves creative recombination of existing resources, building coalitions to access needed capabilities, and demonstrating resourcefulness in overcoming constraints (Baker and Nelson, 2005). It also encompasses translating opportunities into plans, designs, or proposals; developing feasible pathways for implementation; and organizing action sequences. Building on this, we treat “planning” and “bricolage” as complementary modes of resource mobilization: one more formal and forward-looking, the other more improvised and constraint-driven.

\textbf{Execution, Implementation, and Active Behavior.} Intrapreneurial tasks involve implementing initiatives despite incomplete information, ambiguous outcomes, and potential resistance (McMullen and Shepherd, 2006). This requires tolerance for ambiguity, adaptive decision-making, and the ability to navigate organizational politics. Unlike routine execution following established procedures, intrapreneurial implementation involves creating new pathways, coordinating cross-functional efforts, and persisting through setbacks. Here we focus on active behaviors that push ideas into practice: launching new products or services, piloting new processes, or establishing new units or outlets.

\textbf{Innovation and Risk-Taking Behaviors.} Central to intrapreneurship is the generation and implementation of novel solutions (Elert and Stenkula, 2020). In this thesis, we interpret this broadly to include solutions ranging from radical innovations that disrupt existing paradigms to incremental improvements that enhance efficiency. Innovation-oriented intrapreneurial behavior includes creative problem framing, out-of-the-box thinking, voicing entrepreneurial ideas, and a willingness to challenge conventional approaches. It is typically coupled with risk-taking: acting under uncertainty, accepting potential downside in pursuit of a “big win,” and, when appropriate, acting first and seeking approval later. This criterion distinguishes intrapreneurial tasks from routine modifications or simple customization.

\textbf{Managerial and Leadership Roles.} Intrapreneurial behavior often involves informal leadership (championing ideas, building support, coordinating cross-functional efforts) regardless of formal position (Kuratko et al., 2005). This includes evangelizing a vision, managing stakeholder relationships, and orchestrating implementation efforts. Building on Burgelman’s (1983) and Kuratko et al.’s (2005) accounts of top, middle-, and operating-level roles, we distinguish three role-specific managerial patterns in our framework: senior-level ratifying and directing roles (V.A), middle-level proposing and shepherding roles (V.B), and first-line experimenting and adjusting roles (V.C). The ability to influence beyond formal authority becomes crucial for intrapreneurial success, particularly for middle-level managers.

\textbf{Eco-Innovation and Environmental Performance.} A growing body of work treats environmentally oriented innovation as a distinct but closely related domain of intrapreneurial behavior. Eco-innovation typically refers to new or improved products, processes, or organizational practices that reduce environmental harm or resource use over the life cycle, for example through lower emissions, waste, energy, or material inputs, and improved recyclability and reusability (Elliott et al., 2021). In our context, we treat tasks as intrapreneurial under Criterion VI when they explicitly involve environmental or sustainability objectives.

In our framework, this criterion captures intrapreneurial efforts that advance organizational adaptation along environmental and sustainability dimensions, even when the underlying business model or core product line remains unchanged.

\textbf{Strategic Renewal and Transformation (background process).} Beyond individual innovations, intrapreneurship contributes to organizational renewal through new strategic directions, business model innovation, or capability development (Burgelman, 1983). This systemic perspective, emphasized here, distinguishes intrapreneurship from isolated improvements and emphasizes contributions to long‑term organizational adaptation and competitiveness. The process through which entrepreneurial ventures emerge follows distinct stages, as Bhave (1994) shows through analysis of 27 ventures. The model encompasses opportunity recognition (internally or externally stimulated), technology setup and organization creation, and exchange with markets, each stage presenting unique challenges where entrepreneurs introduce varying amounts of novelty. When applied to intrapreneurship within existing organizations, this process perspective suggests that success depends not only on identifying opportunities but also on navigating the iterative, nonlinear, and feedback‑driven nature of venture creation under organizational constraints. In our operationalization, aspects of strategic renewal are captured indirectly whenever task descriptions involve new business activities, products, processes, or organizational practices under Criteria I–III and VI; we therefore do not treat “strategic renewal” as a separate coding criterion, but as an overarching process dimension that these criteria jointly express.

\textbf{Multi-dimensional construct validation.} Our criteria build on Antoncic and Hisrich’s (2001) refinement of intrapreneurship into four distinct but related firm-level dimensions: new business venturing, innovativeness, self-renewal, and proactiveness (the latter encompassing initiative, risk-taking, and competitive aggressiveness). Using data from Slovenian and U.S. firms, they integrate the ENTRESCALE and Zahra’s corporate entrepreneurship scale into a four-dimensional intrapreneurship construct and report moderately good cross-culturally generalizable convergent and discriminant validity, as well as acceptable nomological validity in terms of expected relationships with organizational and environmental antecedents and with firm performance outcomes, including growth and profitability. We borrow from this evidence the idea that intrapreneurship is inherently multidimensional rather than unidimensional, and we operationalize that at the task level by allowing our coding criteria to co-occur rather than enforcing mutual exclusivity.

\textbf{Study-specific operationalization.} We consolidate these theoretical strands into the six criteria presented above as a study-specific, inclusive-OR operationalization suitable for analyzing O*NET task descriptions. The criteria span opportunity discovery, planning, preparation, and advocacy, execution, implementation, and active behavior, innovative and risk-taking behaviors, role-specific managerial actions, and eco-innovation / environmental performance. A task qualifies as intrapreneurial if it satisfies any of the six criteria, reflecting the theoretical understanding that different forms of intrapreneurial behavior may appear in different occupational contexts. This operationalization represents our working theory rather than claiming to be the definitive or only legitimate decomposition of intrapreneurship.

\subsubsection{Human Capital in Innovation Management}

The relationship between human capital and intrapreneurship has evolved from education-based conceptualizations toward more nuanced skill and task-based frameworks. Human capital theory treats schooling and training as investments that raise workers' productivity (and, in competitive markets, their wages) (Becker, 1964). Recent work on intrapreneurial skills (e.g., van Wetten et al., 2020) shows that intrapreneurial capability involves specific competencies that may not be well captured by formal educational attainment.

Contemporary research distinguishes between general human capital (transferable across contexts) and specific human capital (valuable within particular organizational or technological domains). Intrapreneurial competencies appear to combine both: general creative and analytical capabilities with context-specific knowledge of organizational resources, processes, and politics (Vargas-Halabí et al., 2017). This dual requirement complicates both selection and development of intrapreneurial talent.

The shift toward task-based frameworks reflects recognition that human capital manifests through specific work activities rather than residing abstractly in individuals. Tasks requiring non-routine cognitive skills, complex communication, and creative problem-solving provide opportunities for intrapreneurial behavior, while routine, procedural tasks offer limited scope for innovation (Autor et al., 2003). This task-level perspective enables more precise identification of where intrapreneurial potential exists within occupational structures.

Section~\ref{sec:ch2-task-structure} develops this shift in more detail using task-based occupational data (Autor et al., 2003; Fonseca et al., 2019).

\subsection{Task Structure and Occupational Positioning of Innovation Work}
\label{sec:ch2-task-structure}

\subsubsection{From Education to Tasks: The Evolution of Human Capital Frameworks}

Classic evidence on schooling–earnings relationships shows that education explains part, but not all, of productivity variation (Mincer, 1974). A task-based approach (Autor et al., 2003) refines where those investments surface in actual work by analyzing what people do on the job. Building on this shift, scholars have argued for contextual and process perspectives on entrepreneurial behavior (e.g., Ucbasaran et al., 2001) rather than relying solely on static proxies such as formal credentials.

The task‑based approach, as introduced by Autor et al. (2003), categorizes work activities along dimensions of routine versus non‑routine and manual versus cognitive. This framework highlights that many highly educated workers perform routine cognitive tasks with limited innovation potential, while some workers without advanced degrees engage in complex problem‑solving and creative activities. The decoupling of education from task content underscores the importance of examining actual work activities rather than formal qualifications. Using Portuguese firm‑level data, Fonseca, de Faria, and Lima (2019) report that a higher share of employees in abstract tasks ("abstractism") is associated with a greater likelihood of introducing innovations and that innovation performance exhibits an inverted‑U relationship with abstractism, suggesting that an intermediate mix of abstract and more routine tasks may be most beneficial.

O*NET's comprehensive task database enables granularity in analyzing occupational content. By documenting specific tasks, work activities, and contextual factors for hundreds of occupations, O*NET provides the basis for understanding where innovation-oriented work occurs (Handel, 2016). This task-level detail reveals substantial within-occupation variation. Some tasks within an occupation may be highly innovative while others remain routine, challenging monolithic characterizations of jobs as either creative or standardized.

\subsubsection{The O*NET Framework and Importance × Frequency Quadrants}

The O*NET Content Model structures occupational information hierarchically, from broad descriptor domains to specific task statements. Tasks represent the finest level of detail, describing discrete work activities performed within occupations. Work Activities aggregate related tasks into broader categories like "Thinking Creatively" or "Processing Information." Work Context captures environmental and organizational factors that shape how tasks are performed (Peterson et al., 2001). For guidance on use of O*NET and interpretive caveats when moving from descriptors to inferences about work, see Williamson (2024).

Two critical dimensions for understanding task positioning are Importance and Frequency. Importance ratings (1-5 scale) indicate how central a task is to successful job performance: its criticality for meeting role requirements. Frequency ratings (1-7 scale from yearly to hourly) capture how often the task is typically performed. These dimensions are conceptually distinct: a task might be performed rarely but remain crucial when needed (e.g., crisis management), or performed frequently but be relatively peripheral to core value creation. In O*NET’s two-part item format, respondents first rate the importance of a descriptor to their job; if it is at least “somewhat important,” they then rate the level required (Handel, 2016). This ordering emphasizes importance to the occupation before the depth of the requirement.

The Importance × Frequency (I×F) quadrant framework divides tasks into four categories based on threshold values (typically 4.0 for Importance, 4.0 for Frequency). Core tasks (high importance, high frequency) represent the routine center of occupational practice: activities performed regularly that define role success. Critical tasks (high importance, low frequency) capture episodic but consequential work. Operational tasks (low importance, high frequency) involve regular but peripheral activities. Peripheral tasks (low importance, low frequency) sit at the margins of occupational practice. Terminology note: these Core/Critical/Operational/Peripheral labels refer to IF‑based quadrants and are distinct from O*NET’s separate “Core/Supplemental” task flags (defined by relevance ≥67\% and mean importance ≥3.0 without using frequency).

This quadrant structure reveals potential tensions in how innovation work is valued. O*NET's importance ratings reflect "importance to the occupation" (how central a task is to routine role performance) rather than strategic organizational value. Under conditions of intense international competition, “high performance workplaces” require new technologies, workplace structures, and skills, yet O*NET’s content model gives only weak coverage of these domains and omits employee-involvement practices that are closely tied to organizational performance (Handel, 2016).

\subsubsection{Task Positioning and the Paradox of Strategic Work}

Yang and DiBenigno (2024) show that frontline staff can implement bursts of incremental change during punctuations (when a jolt temporarily loosens constraints and increases managerial receptivity) rather than change unfolding only as slow, continuous improvement. In our framework, we interpret such bursts as episodic innovative activities that may be infrequent yet consequential, which helps explain why some high‑impact innovation tasks could appear low‑frequency in task data and cluster in our Critical (high‑importance/low‑frequency) and Peripheral (low‑importance/low‑frequency) quadrants.

The concept of occupational centrality versus strategic importance illuminates why intrapreneurial tasks might cluster in peripheral quadrants. Tasks central to occupational identity and daily practice (processing transactions, maintaining equipment, serving customers) achieve higher importance ratings because they define what practitioners "do" in their roles. Because O*NET's two-part items explicitly ask respondents how important a characteristic is for their job (and then, if at least somewhat important, the level required) (Handel, 2016), day-to-day tasks can receive higher "importance" scores than episodic but strategically significant innovation work. This is a scope condition worth remembering when interpreting quadrant placements.

As we later show in Section~\ref{sec:work-activities}, intrapreneurial work is enriched in "Thinking Creatively," "Developing Objectives and Strategies," and "Selling or Influencing Others," and comparatively depleted in "Documenting/Recording Information" and "Processing Information." These patterns are consistent with the idea that innovation work involves distinct cognitive and social processes rather than simply enhanced versions of routine activities.

\subsubsection{Task Structure and Innovation Performance}

Balancing exploration and exploitation draws on flexible attention control rather than a simple monotonic relationship. Neurocognitive evidence suggests that individuals' ability to shift attentional control states supports effective transitions between exploration and exploitation, which in turn relates to decision-making performance (Laureiro-Martínez et al., 2015). This implies that arranging intrapreneurial tasks to allow periodic shifts between focused execution and broader search can facilitate idea recombination without fragmenting attention.

The concept of "abstract" versus "concrete" tasks provides another lens for understanding innovation potential. Abstract tasks involve conceptualization, pattern recognition, and mental manipulation of ideas, cognitive processes difficult to codify or automate. Concrete tasks involve physical manipulation or rule-based processing amenable to standardization. Fonseca, de Faria, and Lima (2019) report that higher levels of task abstractism, measured as the share of abstract tasks in a firm's workforce, are associated with greater innovation propensity. Their analysis of Portuguese firms reveals that the optimal organizational task structure combines abstract cognitive work with complementary routine tasks that provide stability and operational efficiency, suggesting that intrapreneurial behavior may emerge more readily in cognitively complex but balanced work environments.

\subsection{Human Agency, Autonomy, and AI-Augmented Work}
\label{sec:ch2-agency}

\subsubsection{Conceptual Distinctions: Autonomy, Agency, and Control}

The concepts of autonomy, agency, and control, while related, capture distinct aspects of work design relevant to intrapreneurship. Autonomy represents the degree of decision latitude afforded by job structures (the formal or informal freedom to determine work methods, scheduling, and priorities) (Kubicek, Paškvan, and Bunner, 2017). This environmental affordance creates space for discretionary behavior but does not guarantee its exercise. Workers with high autonomy may still choose routine approaches if they lack motivation, capability, or organizational support for innovation.

Agency, by contrast, refers to the actual exercise of independent judgment and decision-making in work performance. It manifests through choices about how to approach problems, when to deviate from standard procedures, and whether to pursue novel solutions. The distinction matters because technological augmentation may preserve formal autonomy while subtly constraining agency through algorithmic recommendations, automated workflows, or surveillance systems that discourage deviation from optimal paths.

Control introduces questions of power and influence over work processes and outcomes. Intrapreneurial behavior often involves attempting to expand one's sphere of control (securing resources, building coalitions, influencing strategic directions) beyond formal authority. This political dimension of intrapreneurship intersects with but differs from both autonomy (structural freedom) and agency (behavioral independence).

\subsubsection{Job Autonomy and Intrapreneurial Behavior}

Research consistently links job autonomy with innovative behavior, though the relationship proves more complex than simple linear correlation. De Spiegelaere, Van Gyes, and Van Hootegem (2016) distinguish four dimensions of job autonomy: work method autonomy, work scheduling autonomy, work time autonomy, and work location autonomy, and demonstrate their differential effects on innovative work behavior. They show that the autonomy dimensions do not contribute equally to innovative work behaviour: work‑method autonomy has the most robust positive association with IWB (innovative work behavior), while other forms such as scheduling, time, and location autonomy show weaker or more context‑dependent relationships. In parallel, the "bright and dark sides" perspective (Kubicek, Paškvan, and Bunner, 2017) clarifies conceptually how different forms and levels of autonomy can either enable experimentation or create overload and ambiguity. This multidimensional perspective suggests that not all forms of autonomy equally enable intrapreneurship.

The mechanisms linking autonomy to intrapreneurship include both motivational and cognitive pathways. Motivationally, autonomy operates through SDT's three basic psychological needs (autonomy, competence, and relatedness), and fosters autonomous (versus controlled) motivation that supports creative engagement (Deci and Ryan, 2000). Cognitively, autonomy enables experimentation, allowing workers to test novel approaches and learn from failures without immediate sanctions. This "psychological safety" for experimentation appears crucial for sustained innovative behavior.

However, excessive autonomy without structure or support can inhibit innovation through decision paralysis or lack of direction. The concept of "guided autonomy" (freedom within boundaries) emerges as potentially optimal for intrapreneurship. Clear strategic priorities and resource constraints may paradoxically enhance creative problem-solving by providing focus and forcing innovative solutions within limitations (Acar et al., 2018).

\subsubsection{AI, Automation, and the Future of Work}

The advent of artificial intelligence fundamentally alters the landscape for intrapreneurial work, raising questions about which innovative capabilities remain uniquely human. The distinction between automation (complete replacement of human tasks) and augmentation (enhancement of human capabilities) proves critical for understanding the future of intrapreneurial roles. While AI excels at pattern recognition, optimization, and prediction within defined parameters, intrapreneurship often involves redefining problems, challenging assumptions, and navigating ambiguous organizational contexts.

The emergence of collaborative intelligence between AI systems and human workers signals an important shift in how organizations approach innovation and value creation. Chowdhury et al. (2022) integrate the knowledge-based view, socio-technical systems theory, and organizational socialization framework to argue that effective AI-human partnership depends on knowledge sharing, employees' AI skills, trust, and role clarity. Their analysis of creative industries (where human creativity has traditionally been unchallenged) suggests that collaborative intelligence capabilities require deliberate organizational development rather than emerging spontaneously from technology deployment. This finding is particularly relevant for intrapreneurial work, where the tacit knowledge and contextual understanding of human workers must be effectively integrated with AI's analytical capabilities. Independent benchmarks (e.g., APEX) show domain-uneven AI productivity gains (Vidgen et al., 2025), which is consistent with our later finding that intrapreneurial tasks cluster in augmentation rather than substitution zones. APEX (arXiv) evaluates tasks in investment banking, consulting, law, and primary care medicine and reports sizable model–human gaps across these domains.

The Human Agency Scale (HAS) framework, developed by Shao et al. (2025) as part of the WORKBank database project, provides a structured approach to understanding human-AI collaboration requirements across five levels. H1: AI handles the task entirely. H2: AI handles the task with minimal human input (e.g., brief verification). H3 represents equal human–AI partnership. H4 represents human drives task completion with AI assistance and supervision of AI output. H5 designates exclusively human tasks where AI provides no meaningful assistance.

The WORKBank database integrates these human agency assessments with O*NET task descriptions, capturing both worker preferences for automation and expert evaluations of technical feasibility across thousands of occupational tasks (Shao et al., 2025). This comprehensive dataset documents a notable gap between worker preferences for automation support and expert assessments of technical feasibility. Workers often desire AI assistance for tedious or stressful tasks that experts judge technically challenging due to contextual complexity or tacit knowledge requirements. Conversely, workers may resist automation of tasks they find engaging or identity-defining even when technical capability exists. This misalignment, documented in the WORKBank analysis, has important implications for workforce planning and technology adoption strategies.

\subsubsection{Implications for Intrapreneurial Work}

The automation potential of intrapreneurial tasks appears limited by several factors. First, opportunity recognition requires understanding subtle contextual cues, stakeholder needs, and strategic implications that current AI struggles to integrate. Second, resource mobilization involves political navigation and relationship building that remain fundamentally human. Third, managing innovation under uncertainty requires judgment about acceptable risks and trade-offs that reflect organizational values and ethics difficult to codify.

The unique and inimitable nature of human creativity in the digital economy may present a substantial barrier to full automation of intrapreneurial work. Holford (2019) argues that current algorithmic approaches, while excelling at efficiency and analytical tasks, fail to capture the complex, ambiguous, and constantly emerging aspects of human creativity and its associated tacit knowledge. Drawing on the distinction between symbolic meaning and algorithmic efficiency, the analysis suggests that purely analytical AI approaches cannot replicate the holistic, intuitive, and meaning-making dimensions of human creative work. This perspective suggests that the future of intrapreneurial work lies not in replacement but in human-centric organizations where technology augments rather than supplants human creative capabilities, particularly in tasks requiring the synthesis of ambiguous information, ethical judgment, and the navigation of complex social dynamics.

However, AI can augment intrapreneurial capabilities in specific ways. Machine learning can identify patterns in large datasets that humans might miss, potentially surfacing innovation opportunities. Natural language processing can scan vast literatures and patents to identify technological adjacencies. Predictive analytics can assess market potential and optimize resource allocation. The key lies in designing human-AI collaboration that leverages computational power while preserving human judgment, creativity, and contextual understanding.

As we show in Section~\ref{sec:human-agency}, intrapreneurial tasks tend to concentrate in the H3 (equal human–AI partnership) and H4 (human-driven with AI assistance) bands of the Human Agency Scale, suggesting that effective augmentation is likely to require sophisticated integration rather than simple substitution. Designing systems that support rather than supplant human creativity, preserve meaningful decision authority, and enhance rather than diminish intrinsic motivation represents an important challenge for organizations seeking to maintain innovative capacity.

Perceived uncertainty is not merely a technological parameter but a situated human construct. We explicitly measure involved uncertainty from both workers and experts to contrast in-the-moment experience with normalized professional assessment. As we show in Section~\ref{sec:uncertainty}, workers systematically report higher involved uncertainty than experts, with the largest gaps where work is episodic or weakly embedded in routines (Critical/Peripheral). This aligns with WORKBank's profile (H3/H4) in which humans retain judgment.

\subsection{Measurement and Operationalization of Intrapreneurship}
\label{sec:ch2-measurement}

\subsubsection{The Measurement Challenge}

The measurement of intrapreneurship presents persistent methodological challenges that have limited theoretical advancement and practical application. Foremost among these is a lack of conceptual clarity and consistency across studies, with researchers using varied definitions, dimensions, and indicators. These problems mirror broader concerns about construct clarity in management research (Suddaby, 2010). Terms like corporate entrepreneurship, intrapreneurship, entrepreneurial orientation, and innovative behavior are sometimes used interchangeably despite representing distinct constructs at different levels of analysis. This terminological confusion complicates meta-analysis and knowledge accumulation.

The extent of this conceptual confusion is documented by Hernández-Perlines (2022), whose comprehensive bibliometric review reveals the field's fragmentation. Using VOSviewer software to map the intellectual structure of intrapreneurship research, the analysis identifies multiple competing theoretical frameworks and terminological variations that have emerged over four decades. This systematic review indicates that the lack of coherent definition has not only hampered theoretical development but also created practical challenges for organizations attempting to implement intrapreneurial initiatives. The study's identification of main author clusters and research streams provides a roadmap for consolidating disparate approaches into a more unified framework.

The predominance of self-report measures introduces significant validity threats. Podsakoff et al. (2003) define common method variance as variance attributable to the measurement method rather than the constructs of interest, which can distort observed relationships, sometimes inflating and sometimes attenuating them. Social desirability bias leads respondents to overstate innovative behaviors that are organizationally valued. Retrospective bias affects recall of past intrapreneurial activities, while attribution bias influences how respondents interpret their role in innovation outcomes.

The choice between trait-based and behavior-based measurement approaches represents another fundamental tension. Trait approaches assess stable individual characteristics like risk propensity, creativity, or proactivity that presumably predispose intrapreneurial behavior. Behavioral approaches measure specific actions like idea generation, resource acquisition, or project championing. Following this behavioral emphasis (e.g., Gawke et al., 2019), behaviors provide greater specificity and actionability for organizational interventions than broad trait measures.

\subsubsection{Existing Measurement Instruments}

Several validated instruments have emerged to measure intrapreneurship, each with distinct strengths and limitations. Scott and Bruce (1994) developed and tested a path model in which leadership (role expectations and leader–member exchange), work‑group relations, and individual problem‑solving style influence individual innovative behavior directly and indirectly through a perceived climate for innovation. Their study is often cited as an early behavioral operationalization of individual innovative behavior at the person level. Building on such foundations, the Employee Intrapreneurship Scale (EIS) developed by Gawke et al. (2019) measures two dimensions: venture behavior (starting new projects, acquiring resources) and strategic renewal behavior (improving processes, adapting to change). The scale is reported to have good psychometric properties and to predict innovation outcomes, though it relies entirely on self-report.

The Corporate Entrepreneurship Assessment Instrument (CEAI) focuses on organizational factors that enable intrapreneurship: management support, work discretion, rewards, time availability, and organizational boundaries (Hornsby et al., 2002). While useful for diagnosing organizational readiness, CEAI measures environmental conditions rather than actual intrapreneurial behavior, limiting its utility for identifying intrapreneurial individuals or tasks.

Competency-based instruments like the Intrapreneurial Competencies Scale (ICS) validate five competencies: opportunity promoter, proactivity, flexibility, drive, and risk-taking (Vargas-Halabí et al., 2017). In this review we discuss "championing" as a complementary construct commonly assessed in the corporate entrepreneurship literature (e.g., Kuratko et al., 2005), not as a distinct ICS factor. The ICS reports that intrapreneurial competencies are distinct from general entrepreneurial skills and predict innovation involvement. However, competency assessments still rely on subjective ratings and may conflate capability with opportunity to exercise skills.

\subsubsection{Objective, Task-Based Approaches to Measurement}

Moving beyond self-reports toward objective measurement represents a critical frontier for intrapreneurship research. Archival measures (patents filed, projects initiated, process improvements documented) provide behavioral indicators less susceptible to response biases. However, these measures often capture outcomes rather than behaviors and may miss failed attempts or informal innovations that nevertheless represent intrapreneurial effort.

Theory-grounded coding of work activities offers a promising alternative. By developing classification criteria derived from theoretical foundations and applying them systematically to task descriptions, researchers can identify intrapreneurial content objectively. This approach requires clear operational definitions, transparent coding procedures, and reliability assessment through multiple coders or repeated application. The key advantage lies in separating the identification of intrapreneurial tasks from individual performance or perceptions.

Large-scale occupational databases enable opportunities for objective measurement. O*NET's task statements, Work Activities, and contextual factors provide standardized descriptions across hundreds of occupations. The European REFLEX survey links educational backgrounds to work tasks across countries. The Global Entrepreneurship Monitor (GEM) tracks entrepreneurial activity internationally. Bosma, Stam, and Wennekers (2012) demonstrate the value of large-scale international studies through their analysis of entrepreneurial employee activity across multiple countries, revealing that intrapreneurship is more prevalent in high-income economies and is negatively correlated with early-stage entrepreneurship at the country level (substitution effects). These large-scale survey datasets enable researchers to measure intrapreneurial employee activity at scale, albeit via self-reported behavior rather than task-level occupational data.



\subsection{Organizational and Individual Determinants of Intrapreneurship}
\label{sec:ch2-determinants}

\subsubsection{Organizational Context and Support}

The Resource-Based View (RBV) of the firm offers a theoretical framework for comprehending how organizational resources can either facilitate or hinder intrapreneurial activities (Barney, 1991).
Barney (1991) posits that only those resources that are valuable, rare, challenging to replicate, and not easily replaceable can yield a sustained competitive edge, assuming the firm is structured to leverage them effectively.
In line with this reasoning, we highlight that competitive advantage is contingent upon the entrepreneurial deployment of resources rather than mere ownership.
From a resource-based perspective (Barney, 1991; Urbano et al., 2013), intrapreneurs are viewed as individuals who identify innovative uses for existing resources and manage their recombination to foster innovation.

Organizational support frameworks significantly impact intrapreneurial endeavors. Research finds that managerial backing and a willingness to accept risk are strongly correlated with innovative outcomes, while human capital also plays a beneficial role (Alpkan et al., 2010). Based on self-determination theory (Deci and Ryan, 2000), Klein and Ben-Hador (2025) present evidence indicating that organizational endorsement of intrapreneurial initiatives is positively linked to employee performance. They propose that employees’ intrapreneurial actions and intra-organizational social capital are crucial factors in this dynamic. Their interpretation aligns with SDT, positing that support for intrapreneurship likely fulfills employees’ fundamental psychological needs for autonomy, competence, and relatedness, thus enhancing the motivation that drives these actions (Deci and Ryan, 2000). Klein (2023) argues that both transformational and transactional leadership styles can be associated with employees’ intrapreneurial activities, partly through the lens of organizational support for entrepreneurship as perceived by employees. He also highlights the intensity of environmental competition as a contextual element that may amplify the connection between transformational leadership and perceived support.

Formal mechanisms such as innovation labs, incubators, and allocated innovation time (for instance, Google's "20\% time") offer institutional credibility and resources for intrapreneurial initiatives. Informal networks and communities of practice are vital for facilitating knowledge exchange and collaboration, which are essential for fostering innovation. Evidence from Badoiu, Segarra-Ciprés, and Escrig-Tena (2020) further supports these conclusions by illustrating how organizational backing from senior management, autonomy in work, and strong relationships with leaders influence daily intrapreneurial activities within a new technology-driven firm.

The relationship between human resource management practices and innovative work behavior has been systematically examined through the ability-motivation-opportunity (AMO) framework. Bos-Nehles, Renkema, and Janssen (2017) conducted a comprehensive review of 27 peer-reviewed articles, identifying the HRM practices that most effectively enhance innovative work behavior (training and development, rewards, job security, autonomy, task composition, job demand, and feedback). This AMO-based literature suggests that organizations should simultaneously develop employees' capabilities (ability), provide appropriate incentives (motivation), and create structural conditions (opportunity) for intrapreneurial behavior to flourish. The framework's emphasis on systemic alignment rather than isolated practices underscores the complexity of designing organizational contexts that support sustained innovation.

Reward systems shape intrapreneurial motivation through both extrinsic and intrinsic mechanisms (Deci and Ryan, 2000; Kuratko et al., 2005). Financial incentives for innovation (bonuses, profit sharing, equity participation) provide tangible recognition but may crowd out intrinsic motivation if they are perceived as controlling rather than autonomy-supportive (Deci and Ryan, 2000; Bos-Nehles et al., 2017). Non-monetary rewards such as autonomy, project ownership, and career advancement opportunities are argued to be more conducive to sustaining long-term innovative behavior because they support employees’ intrinsic motivation and sense of ownership over ideas (Deci and Ryan, 2000; Bos-Nehles et al., 2017; Kuratko et al., 2005). The challenge lies in balancing rewards for successful innovation with explicit tolerance for intelligent failures that generate learning and signal continued support for entrepreneurial effort (Kuratko et al., 2005).

Leadership styles differentially affect intrapreneurship. Building on this line of work, Klein (2023) and others suggest that transformational and transactional leadership can be positively related to employees’ intrapreneurial behaviors, with perceived organizational support for entrepreneurship acting as an important intervening mechanism and external competition as a relevant boundary condition. Transformational leaders articulate a compelling vision, challenge followers’ assumptions, and empower them to pursue creative and innovative solutions by questioning assumptions, reframing problems, soliciting new ideas, and avoiding public criticism of mistakes, while also signalling a willingness to take risks themselves (Bass and Riggio, 2006). By fostering this supportive climate for innovation, they enhance employees’ perceptions that innovation is expected and resourced, which in turn predicts higher levels of innovative behavior (Scott and Bruce, 1994).  In contrast, transactional leadership, particularly management-by-exception, focuses on clarifying standards, monitoring deviations, and correcting errors, which supports compliance and efficiency but is less conducive to innovation than the transformational components, which augment transactional leadership in predicting outcomes such as innovation, risk taking, and creativity (Bass and Riggio, 2006).

\subsubsection{Employee Roles and Hierarchical Dynamics}

The relationship between hierarchical position and intrapreneurial behavior is complex and shaped by organizational context. Kuratko et al. (2005) conceptualize corporate entrepreneurship as a multi-level phenomenon in which top-level managers articulate an entrepreneurial mission, design organizational architecture, and allocate resources, while middle-level managers occupy a bridging role by endorsing, refining, and shepherding entrepreneurial opportunities and by identifying, acquiring, and deploying resources to pursue them. Hornsby et al. (2002) show that middle managers’ perceptions of management support, work discretion, rewards and reinforcement, time availability, and organizational boundaries strongly influence their willingness to support and engage in entrepreneurial initiatives.

Extending this view, Gawke et al. (2019) argue that senior managers, middle managers, first-level managers, and non-managerial employees all play complementary roles in new venture creation and strategic renewal: senior-level managers create the organizational vision and architecture and rationalize how new businesses and strategic choices add value; middle-level managers champion, refine, and channel bottom-up ideas upward and translate top-down intrapreneurship strategies for primary implementers; first-level managers and their teams operationalize and experiment with provided resources to exploit opportunities; and non-managerial employees may deviate from formal work requirements to generate and nurture innovative ideas before formally revealing them to management (Gawke et al., 2019).

Role expectations and formal job structures also influence how easily employees can act intrapreneurially. Work on intrapreneurial employees shows that factors such as developmental support, coaching, trust in supervisors, and horizontal participation stimulate opportunity identification and involvement in innovation projects, indicating that contextual support legitimizes time and effort devoted to such activities (Blanka, 2018; Neessen et al., 2018).

At the same time, research on bootlegging suggests that strategic autonomy, rewards for innovation accomplishments, and intrapreneurial self-efficacy encourage employees to bypass official channels and commit their own resources, whereas high levels of front-end formality in idea exploration, development, and selection reduce bootlegging, implying that overly rigid formal mechanisms can dampen bottom-up experimentation even when some structure is needed to guide behavior (Globocnik and Salomo, 2014; see also Blanka, 2018). 

Pinchot (1985) similarly recommends that intrapreneurial venture teams operate in a “freewheeling” atmosphere with quick decision processes and deliberately vague job descriptions, noting that most intrapreneurs dislike tightly specified roles. Cross-disciplinary work on creativity under constraints converges on a related conclusion: constraints such as rules, procedures, and resource limits can foster innovation when they provide focus and challenge, but overly restrictive constraints tend to suppress creative problem solving, implying that clear innovation expectations are most effective when combined with autonomy and flexible role definitions rather than highly prescriptive innovation mandates (Acar et al., 2018).

\subsubsection{Occupational and Task-Level Deployment of Intrapreneurial Skills and Their Consequences}

Occupational variation in the utilisation of intrapreneurial skills shows systematic patterns. Using REFLEX graduate data, Bjornali and Støren (2012) show that graduates in engineering and science are more likely than those in business and administration to report having introduced innovations at work, reflecting the more technology-oriented task content of many STEM jobs. Building on the same dataset, van Wetten et al. (2020) distinguish STEM and business occupations and show that, in both groups, involvement in product, service and process innovation is strongly associated with intrapreneurial skills, with creativity and brokering particularly salient in STEM roles and championing skills especially important in management and sales jobs. In contrast, business and management roles, particularly in management and sales, contribute to innovation through activities such as recognising opportunities, mobilising resources and championing new ideas, which draw heavily on brokering and championing competencies and correspond to intrapreneurial roles such as knowledge broker and innovation champion (Hayton and Kelley, 2006).

Within any given occupation, however, employees may differ markedly in the extent to which their day-to-day work involves routine tasks versus creative problem-solving and change-oriented responsibilities. Job crafting research shows that employees proactively alter their tasks, relationships and resources and that such task-level crafting varies between individuals in the same job, and has been linked to higher engagement, creativity and extra-role behaviour (Demerouti and Bakker, 2013). Studies of knowledge brokering and front-end innovation similarly document boundary-spanning, role-based activities such as translation, enrolment and pitching that cut across formal job titles and strongly influence which ideas progress (Hargadon, 1998; Jensen et al., 2018). Digital-trace analyses of workplace communication further show that employees who play strong roles in innovation often occupy boundary-spanning positions in organisational networks, even when they share the same occupational label as less innovative colleagues (Gloor et al., 2020). Collectively, this evidence indicates that analyses focusing on specific task assignments, emergent roles and network positions provide a more precise picture of how intrapreneurial skills are deployed in organisations than analyses that rely on occupational titles alone.

Research on skill mismatch, as summarised by van Wetten et al. (2020), generally shows that when workers possess higher skill levels than their jobs require, unused skills tend to depreciate and workers report lower job satisfaction, fewer learning opportunities, lower wages and a higher likelihood of voluntary quits. Conceptualising the underuse of intrapreneurial capabilities as a form of perceived overqualification or ability–job misfit, studies of perceived overqualification generally find that employees who feel their capabilities exceed role requirements report lower job satisfaction and affective organisational commitment and higher turnover intentions, with some evidence that these intentions translate into actual turnover (Erdogan and Bauer, 2021). It is therefore reasonable to expect that chronic underutilisation of intrapreneurial skills not only reduces involvement in innovation but also increases the risk that intrapreneurially capable employees will eventually leave, so that organisations face both opportunity costs from unrealised innovation and direct costs associated with the loss of employees whose capabilities are underused (van Wetten et al., 2020; Erdogan and Bauer, 2021).