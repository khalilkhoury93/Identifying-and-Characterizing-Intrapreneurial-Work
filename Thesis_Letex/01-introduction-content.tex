\section{Introduction}

\subsection{Problem Motivation and Research Gap}

The field of intrapreneurship research has experienced significant growth over the past four decades, yet remains characterized by considerable conceptual fragmentation and methodological diversity. Neessen et al. (2018) synthesize conceptual strands in the field, while Blanka (2018) surveys measurement approaches. This fragmentation challenges theoretical advancement and practical application. Scholars and practitioners struggle to establish consensus on fundamental definitions, measurement approaches, and the mechanisms through which intrapreneurial behavior influences organizational outcomes. Suddaby (2010) underscores the importance of construct clarity. Despite recognition that intrapreneurial behavior, defined as pursuing entrepreneurial opportunities within existing organizations, is critical for organizational adaptation, the field lacks objective, reproducible methods for identifying such work\footnote{This thesis uses three related terms to describe the same phenomenon. "Intrapreneurial tasks" is the primary technical term, referring to tasks that meet at least one of six theoretical criteria (operationalized in Section~\ref{sec:ch3-criteria-dev} and Appendix B.1). "Innovation work" and "innovative work" are used interchangeably as more accessible synonyms. "Strategic innovation work" appears when we specifically discuss the paradox between a task's peripheral position in occupational structures yet vital importance to organizational strategy. All three terms refer to the same set of tasks identified through our classification methodology.}.

The digital transformation of work has elevated the importance of task-level analysis in understanding how innovation emerges within organizations. Traditional approaches focusing on individual traits or organizational characteristics often do not fully capture the granular reality of how intrapreneurial work is actually performed. Handel (2016) discusses both the strengths and limitations of O*NET for this purpose. As artificial intelligence and automation technologies reshape job structures, understanding the specific task-level requirements for innovation becomes critical for workforce planning and development. Williamson (2024) provides an occupational analysis perspective that motivates this shift toward task-based frameworks, building on labor economics research that decomposes jobs into routine versus non-routine and cognitive versus manual activities. We build on this tradition but focus specifically on innovation‑oriented, opportunity‑seeking tasks rather than all non‑routine work.

The challenge of objective measurement has long plagued intrapreneurship research, with many studies relying on self-reported measures that are susceptible to common method bias and social desirability effects. Gawke et al. (2019) argue for theory-grounded, behaviorally anchored approaches. By leveraging large-scale occupational databases and applying structured criteria derived from theoretical foundations, researchers can move beyond subjective assessments toward more reliable identification of intrapreneurial behavior.

Intrapreneurial work appears structurally peripheral in occupational frameworks, compositionally skewed toward discovery and planning, and systematically located in higher-agency regions of the human–AI collaboration spectrum than routine work.

A critical gap exists in understanding where intrapreneurial work sits within occupational structures and how it relates to routine operational activities. O*NET, the U.S. Department of Labor's comprehensive occupational database with task-level descriptions and standardized importance (how critical to doing the job) and frequency (how often the task is performed) ratings, makes it possible to examine this positioning, yet, to our knowledge, prior research has not systematically mapped intrapreneurial tasks onto this Importance × Frequency (IF) space. For interpretive clarity, we label the four IF quadrants as Core (high importance, high frequency), Critical (high importance, low frequency), Operational (low importance, high frequency), and Peripheral (low importance, low frequency). Note that O*NET does not name IF-based quadrants. O*NET’s separate ‘Core/Supplemental’ task flags are defined by relevance (≥67\%) and importance (mean ≥3.0) without using frequency; our ‘Core quadrant’ is therefore distinct from O*NET’s ‘Core tasks’.

This positioning suggests a paradox: if routine work is defined by high importance and high frequency, then innovation work is episodic (performed infrequently) and experimental; it should therefore appear peripheral in occupational metrics even while remaining strategically vital. Consider platform redesign: it happens rarely (low frequency in O*NET terms) yet transforms organizational capabilities when it does occur. As we show in Section~\ref{sec:results}, intrapreneurial tasks appear less frequently than expected in the Core quadrant while appearing more frequently than expected in the Critical and Peripheral quadrants. This pattern is consistent with strategic innovation work operating at occupational edges rather than centers.

Furthermore, as organizations increasingly adopt AI technologies, understanding the human agency requirements of intrapreneurial work becomes essential for designing effective human-AI collaboration systems. The WORKBank database (a large-scale dataset with paired worker and expert judgments on automation potential and human agency for O*NET tasks; Shao et al., 2025) provides a unique lens for examining these requirements, yet the specific characteristics of intrapreneurial tasks within this framework remain underexplored.

\subsection{Research Questions}

This study addresses three interconnected research questions that bridge theoretical understanding with practical application:

\textbf{RQ1: How are intrapreneurial tasks distributed across O*NET Importance × Frequency quadrants?}

\textbf{RQ2: How do worker and expert assessments of human agency and automation potential differ for intrapreneurial versus non-intrapreneurial (“routine”) tasks?}

\textbf{RQ3: Can intrapreneurial tasks be identified reliably using theory-grounded criteria applied to task descriptions?}

\subsubsection*{How the Current Study Addresses These Questions}

Chapter~3 addresses the research questions as follows:

\textbf{RQ1: Task-Level Positioning} - Analyzing how intrapreneurial tasks distribute across the Importance × Frequency quadrant framework helps characterize the occupational positioning of innovation work. This addresses the centrality–value tension by testing whether strategic innovation tasks cluster in core or peripheral quadrants, with implications for how organizations structure and recognize intrapreneurial contributions.

\textbf{RQ2: Human Agency Profiling} - Applying the Human Agency Scale to compare intrapreneurial and routine (non-intrapreneurial) tasks provides evidence on automation potential and augmentation requirements. By capturing both worker and expert perspectives, the analysis examines the desire–capability gap and its implications for the future of innovation work. In addition, we conduct an uncertainty calibration analysis by pairing worker and expert involved-uncertainty ratings at the task level and testing their alignment within Importance × Frequency quadrants. We further assess the WBU (Willingness to Bear Uncertainty) index as an independent indicator of uncertainty‑driven agency.

\textbf{RQ3: Objective Operationalization} - By developing theory-grounded criteria derived from the six dimensions identified in Section~\ref{sec:ch2-foundations} and applying them systematically to O*NET task descriptions, the study moves beyond self-report toward reproducible classification. The use of structured prompts, multiple independent classifications, and consensus procedures enhances reliability while maintaining theoretical fidelity.

The WORKBank database, with its integration of O*NET task descriptions, worker assessments, and expert evaluations, provides granularity for addressing these questions. By combining multiple data sources at the task level (the fundamental unit of work), this approach bridges the gap between abstract conceptualization and practical application.

\subsection{Research Contributions}

This research contributes across theoretical, methodological, and practical dimensions:

\subsubsection{Theoretical Contributions}

First, the study suggests that innovation work operates according to different temporal and importance dynamics than routine work. The finding that intrapreneurial tasks cluster in Critical and Peripheral quadrants while being depleted in Core positions highlights a tension between occupational centrality and strategic value for the organization as a whole.

Second, the research contributes to human capital theory by supporting a shift from education-based to task-based frameworks for understanding innovation capability. We find that the correlation between importance and frequency is substantially weaker for intrapreneurial tasks than for routine tasks (see Section~\ref{sec:if-coupling}; Appendix C.4), and that strategic work may follow different organizing principles than operational work, with implications for how organizations develop and deploy human capital.

\subsubsection{Measurement Contributions}

Theory-grounded structured criteria enable reproducible, objective classification at scale. The high agreement rate across independent classification runs demonstrates that theoretical concepts can be successfully operationalized into stable, reliable measures. This approach enables objective assessment of complex organizational behaviors that have traditionally resisted measurement.

Category tags used in secondary analyses (I–IV, VI, and V.A/V.B/V.C) are derived from the model’s output produced by the structured prompt (Appendix~\ref{app:B1}), using the mapping described in Sections~\ref{sec:ch3-category-framework}–\ref{sec:ch3-phenotype-derivation}, and are used to compute category prevalence and derive phenotypes (Section~\ref{sec:internal-structure}; Appendix~\ref{app:F}).

The integration of multiple perspectives (theoretical classification, occupational positioning, worker assessments, and expert evaluations) creates a multi-dimensional framework for understanding intrapreneurial work. This triangulation reveals insights unavailable from single approaches, such as systematic gaps between worker and expert uncertainty perceptions.

\subsubsection{Design Implications}

For AI system design, the concentration of intrapreneurial tasks in the middle-to-upper bands of the Human Agency Scale, specifically H3 (equal human–AI partnership) and H4 (human-driven task completion with AI assistance), from a five-point scale where H1 represents full automation and H5 represents exclusively human work, suggests that augmentation rather than automation is likely to be more effective as a primary focus. The scarcity of tasks rated as requiring exclusively human performance (H5) suggests that even highly creative work is seen as amenable to some form of technological support, though design must preserve human agency and judgment.

Moving from theory to practical application, the findings challenge traditional performance evaluation and career development approaches emphasizing routine excellence. Organizations may need to develop new frameworks that recognize and reward episodic but strategically critical innovation work, even when such work appears peripheral from an occupational perspective. The gap we identify between worker and expert assessments in the Peripheral quadrant suggests opportunities for targeted technology development.

\subsubsection{Innovation Statement}

This study's approach represents a methodological advance over existing intrapreneurship measurement approaches. Classic self-report measures are susceptible to social desirability bias and common method variance. Organizational climate tools like the Corporate Entrepreneurship Assessment Instrument (CEAI) assess environmental conditions but not actual intrapreneurial behavior. Archival outcome proxies (patents, new products) capture results but miss failed attempts and informal innovations. In contrast, our approach uses large language models to systematically apply six theory-grounded criteria to hundreds of O*NET task descriptions, with multiple independent classification runs achieving high inter-run agreement. We prompt the model to apply these criteria to each task three times and aggregate by majority vote to support stability. This enables more objective, reproducible identification of intrapreneurial work at scale, paired with dual-perspective human agency assessments from both workers performing the tasks and domain experts evaluating automation feasibility.

\subsection{Method Overview}

This research combines computational and data-driven methods. We develop structured criteria from intrapreneurship theory, operationalizing six core dimensions: (I) opportunity discovery and idea generation; (II) planning, preparation, and advocacy (including resource mobilization and internal issue selling); (III) execution, implementation, and active behavior in new business activities; (IV) innovative and risk-taking behaviors; (V) role-specific managerial tasks at senior, middle, and first levels; and (VI) eco-innovation and environmental performance. These criteria are encoded into prompts that guide a large language model to classify each task description as intrapreneurial or not, with each task undergoing multiple independent classification runs to support reliability. We apply an inclusive‑OR rule: a task is labeled intrapreneurial if it satisfies at least one of the six criteria; majority voting across runs ensures label stability.

The classified tasks are then analyzed through O*NET's Importance × Frequency (IF) framework, dividing the task landscape into four quadrants based on thresholds of 4.0 for both dimensions. This quadrant analysis helps reveal where intrapreneurial work sits within occupational structures. The analysis is further enriched by integrating worker and expert assessments from the WORKBank database, which provides Human Agency Scale ratings, automation desire and capability assessments, and perceived uncertainty measures for each task.

Analytically, we use established non‑parametric tests with multiple‑testing correction and report appropriate effect sizes; full procedures and robustness checks appear in Section~\ref{sec:ch3}.

\subsection{Thesis Structure}

Following this introduction, Chapter~\ref{sec:ch2} presents the Literature Review, establishing the theoretical foundations of intrapreneurship and examining the evolution from education-based to task-based frameworks for understanding human capital. The literature review synthesizes research on intrapreneurial behavior, task structure, human agency requirements, and measurement approaches, identifying persistent gaps that motivate the investigation.

Chapter~\ref{sec:ch3} details the Methods and Data, describing the WORKBank database provenance, sample characteristics, and analytical procedures. This chapter details the label adjudication process using structured criteria, the construction of IF quadrants, the Human Agency Scale framework, and statistical methods employed. The chapter also addresses data quality, ethical considerations, and reproducibility provisions.

Chapter~\ref{sec:results} presents the Results and Analysis through comprehensive examination of how intrapreneurial tasks distribute across the occupational landscape, their human agency requirements, and the alignment between worker and expert assessments. The chapter integrates quantitative findings with illustrative examples to demonstrate the distinctive characteristics of intrapreneurial work.

Chapter~\ref{sec:discussion} provides Discussion and Implications, interpreting the findings in light of existing theory and deriving practical recommendations for organizations, technology developers, and policymakers. The chapter addresses how organizations can better support intrapreneurial work through task design, performance systems, and human-AI collaboration frameworks.

Chapter 6 concludes with a synthesis of contributions, acknowledgment of limitations, and directions for future research. Appendices provide technical details, supplementary analyses, and reproducibility materials to support transparency and enable extension of this work.

This thesis examines intrapreneurial work at the task level, bridging theory with practice. By establishing objective methods for identifying innovation work, we contribute to scientific knowledge and organizational practice.
