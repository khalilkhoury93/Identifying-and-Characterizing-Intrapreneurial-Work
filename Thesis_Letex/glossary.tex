\section{Glossary and Abbreviations}\label{glossary-and-abbreviations}

\textbf{Note:} Normative definitions of core constructs
(intrapreneurship, innovation work, strategic innovation work) appear in
Chapter 1. This glossary provides acronym expansions and quick-reference
definitions for convenience.


\textbf{BH-FDR:} Benjamini-Hochberg False Discovery Rate - Statistical
method for controlling false positive rate when conducting multiple
comparisons.

\textbf{CI:} Confidence Interval - Range of values likely to contain the
true population parameter, typically reported at 95\% level.

\textbf{Core (Quadrant):} Tasks with both high importance (≥4.0) and
high frequency (≥4.0) representing routine occupational center
(IF‑based; distinct from O*NET ``Core'' task flag).

\textbf{Critical (Quadrant):} Tasks with high importance (≥4.0) but low
frequency (\textless4.0) representing episodic consequential work.

\textbf{Δ (Delta):} Difference between paired measurements, specifically
worker mean minus expert mean for the same task.

\textbf{H1-H5:} Human Agency Scale levels from H1 (AI handles entirely)
to H5 (exclusively human task).

\textbf{HAS:} Human Agency Scale - Five-level framework (1-5)
characterizing human involvement required when AI assistance is
available.

\textbf{IF:} Importance--Frequency: Two-dimensional framework for
categorizing tasks based on O*NET ratings.

\textbf{Intrapreneurship:} Entrepreneurial behavior within existing organizations that discovers, plans, advocates, and implements opportunities; shows initiative, innovation, and judicious risk-taking; and/or performs role-specific managerial or eco-innovation tasks that advance new business activities.

\textbf{Classification Criteria (I--VI):} Shorthand for the six behavioral criteria used in the LLM-based intrapreneurial task classification: (I) Opportunity Discovery and Idea Generation; (II) Planning, Preparation, and Advocacy; (III) Execution, Implementation, and Active Behavior; (IV) Innovative and Risk-Taking Behaviors; (V) Role-Specific Managerial Tasks; and (VI) Eco-Innovation and Environmental Performance. See Section~\ref{sec:ch3-criteria-dev} and Appendix~\ref{app:B1}.

\textbf{κ (Kappa):} Cohen's kappa coefficient measuring inter-rater
agreement corrected for chance agreement.

\textbf{LLM:} Large Language Model - AI system trained on extensive text
data capable of understanding and generating human-like text.

\textbf{O*NET:} Occupational Information Network - U.S. Department of Labor
database containing standardized occupational information.

\textbf{OR:} Odds Ratio - Measure of association between exposure and
outcome, quantifying effect size in enrichment analyses.

\textbf{Peripheral (Quadrant):} Tasks with both low importance
(\textless4.0) and low frequency (\textless4.0) at occupational margins.

\textbf{q-value:} Adjusted p-value after false discovery rate correction
for multiple testing.

\textbf{ρ (Rho):} Spearman's rank correlation coefficient measuring
monotonic association between variables.

\textbf{Operational (Quadrant):} Tasks with low importance
(\textless4.0) but high frequency (≥4.0) representing regular peripheral
activities.

\textbf{O*NET Core/Supplemental (Flag):} Separate O*NET labeling based on (a)
relevance (≥67\% of respondents) and (b) mean importance (≥3.0).
Frequency is not used. Do not conflate with IF‑based quadrants.

\textbf{SALT Lab:} Stanford Social and Language Technologies - Research
group at Stanford University conducting the WORKBank project.

\textbf{SOC:} Standard Occupational Classification - Federal statistical
standard for classifying workers into occupational categories.

\textbf{STEM:} Science, Technology, Engineering, and Mathematics -
Academic and professional disciplines in technical fields.

\textbf{WA:} Work Activities - O*NET descriptors aggregating related
tasks into broader behavioral categories.

\textbf{WBU:} Willingness to Bear Uncertainty - Index capturing worker
preference for human control when facing uncertain tasks.

\textbf{WC:} Work Context - O*NET descriptors capturing environmental
and organizational factors shaping task performance.

\textbf{WORKBank:} Database developed by Stanford SALT Lab containing
worker and expert assessments of task automation potential.

\textbf{z-score:} Standardized score indicating how many standard
deviations a value is from the mean.









